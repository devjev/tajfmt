\documentclass{article}

\usepackage{tajfmt}

\begin{document}
\section{Special Notation}
We use Euler notation for ordinary and partial derivatives. For example: a derivative of the function $f$ is denoted by $\df[z]{f}$, the differential, however, is denoted by $\dif{f}$. Furthermore, higher order derivatives are denoted as follows: $\ddf[z]{f}$ means ``the second order derivative of $f$ in relation to $z$''.

As an example, consider It\^o's formula in the above notation:
\begin{align*}
	\text{given } \dif{x} &= a\dif{t} + \sigma\dif{W} \\
	\df{F} &= {\Big[a\df{F} + \df[t]{F} + \frac{\sigma^2}{2}\ddf{F}\Big]
		\dif{t} + \sigma\df{F}\dif{W}}
\end{align*}

Or, take Taylor series expansion:
\begin{align*}
	F(x + \delta{}x) &= {F(x) + \df{f}\delta{}x + 
		\frac{\ddf{f}}{2!} \delta{}x^2 + 
		\frac{\dddf{f}}{3!}\delta{}x^3 + \dots}
\end{align*}

Another peculiarity of this notation, is that common parens are \emph{never} used for
grouping expression, but rather only for function application. Expressions are groupped with square parens, like this $[a+b]^2 = a^2 + b^2 + 2ab$.

This gives us additional expressive power in our notation, like for example a suggested short hand notation for derivation:
\begin{align*}
	\ds[t]{f} &= \frac{1}{a^2}\dds{f}
\end{align*}


\section{Techical}

Also, these documents will quite often use a verbatim font, \Verb|something_like_this| to denote computer input or output.

\end{document}